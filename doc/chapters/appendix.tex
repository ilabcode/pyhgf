\section{Appendix}

\subsection{Computations in VOPE coupling}

In the following, we present the computations of volatility nodes according to the proposal that each cortical level of a hierarchy implements its own volatility estimation.\\

The \textsf{Update} steps in volatility coupling are:

\begin{equation}
	\check{\mu}_i^{(k)} = \hat{\check{\mu}}_i^{(k)} + \frac{1}{2} \frac{\gamma_{i}^{(k)}}{\check{\pi}_i^{(k)}} \Delta_{i}^{(k)}
\end{equation}

\begin{equation}
	\check{\pi}_i^{(k)} = \hat{\check{\pi}}_i^{(k)} + \frac{1}{2} (\gamma_{i}^{(k)})^2 + (\gamma_{i}^{(k)})^2 \Delta_{i}^{(k)} - \frac{1}{2} \gamma_{i}^{(k)} \Delta_{i}^{(k)}
\end{equation}

In the \textsf{PE} step, the node computes:

\begin{equation}
  \Delta_i^{(k)} = \frac{\hat{\pi}_i^{(k)}}{\pi_{i}^{(k)}} + \hat{\pi}_i^{(k)} (\delta_i^{(k)})^2 - 1. 
\end{equation}

Finally, In the \textsf{Prediction} step, we now need to compute four nodes: \\
the predicted (volatility) mean

\begin{equation}
	\hat{\check{\mu}}_i^{(k+1)} = \check{\mu}_i^{(k)},
\end{equation}

the precision of that prediction

\begin{equation}
  \hat{\check{\pi}}_i^{(k+1)} = \frac{1}{\frac{1}{\check{\pi}_i^{(k)}} + \nu_i^{(k+1)}}, 
\end{equation}

the predicted environmental uncertainty (as a function of the next higher level in the hierarchy, $\mu_{i+1}$)

\begin{equation}
  \nu_i^{(k+1)} = \exp(\kappa_{i,i+1} \mu_{i+1}^{(k)} + \omega_i),
\end{equation}

and the new (auxiliary) expected precision

\begin{equation}
  \gamma_i^{(k+1)} = \kappa_{i+1,i} \nu_i^{(k+1)} \hat{\pi}_i^{(k+1)}.
\end{equation}

The last node is only defined for convenience in terms of simplifying the equations and the corresponding message passing.


\subsection{Possible PC-like implementation of VOPE coupling}

\loadfigure[fig:voall]{figures/vope_all}
\loadfigure[fig:withvol]{figures/zoom_vape_with_vope}

Figure \ref{fig:voall} displays one proposal for within-column volatility estimation, where we've zoomed in to a level $i$ of the cortical hierarchy (and a value parent $i+1$) and added its volatility parent to the superficial layers. \\

An alternative implementation of this idea is displayed in figure \ref{fig:withvol}. Here, only the precision-related computations of the volatility parent are placed in the superficial layers, while the corresponding prediction errors, $\Delta_i$ and $\check{\delta}_i$ as well as the actual volatility estimate $\check{\mu}_i$ live in middle layers, and the prediction of the volatility estimate, $\hat{\check{\mu}}_i$, lives in layer 6. This setup stresses the structural similarities to the message passing entailed by mean coupling. Moreover, using this setup, we can now easily depict how a higher cortical level $i+1$ would serve as a value parent to the volatility estimate in level $i$ - instead of predicting the mean - by simply exchanging the arrows between the levels as shown in figure \ref{fig:volparent}. 

\loadfigure[fig:volparent]{figures/zoom_vope}

